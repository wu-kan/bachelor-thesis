\chapter{算子编译脚本}

\label{算子编译脚本}

\begin{python}
#!/bin/sh
# -*- python -*-

# This file is bilingual. The following shell code finds our preferred python.
# Following line is a shell no-op, and starts a multi-line Python comment.
# See https://stackoverflow.com/a/47886254
""":"
spack load py-numpy py-decorator py-sympy py-mpi4py
time mpirun -n $1 python3 $0
exit 0
":"""
# Line above is a shell no-op, and ends a python multi-line comment.
# The code above runs this file with our preferred python interpreter.

import numpy  # atc
import decorator  # atc
import sympy  # atc
import glob
from subprocess import getoutput
from mpi4py import MPI

if __name__ == "__main__":

    comm = MPI.COMM_WORLD
    commSize = comm.Get_size()
    commRank = comm.Get_rank()

    singleopList = []

    if commRank == 0:
        for singleop in glob.glob('*.json'):
            if singleop not in singleopList:
                if len(glob.glob(singleop[0:-4]+'om')) == 0:
                    singleopList.append(singleop)

    singleopList = comm.bcast(sorted(singleopList), root=0)

    for i in range(commRank, len(singleopList), commSize):
        log = {}
        log['taskRank'] = i
        log['taskSize'] = len(singleopList)
        log['commRank'] = commRank
        log['commSize'] = commSize
        log['command'] = 'atc --soc_version=Ascend910 --output=. --singleop='+singleopList[i]
        log['output'] = getoutput(log['command'])
        print(log)
\end{python}
