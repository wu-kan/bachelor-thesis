%%
% 摘要信息
% 本文档中前缀"c-"代表中文版字段, 前缀"e-"代表英文版字段
% 摘要内容应概括地反映出本论文的主要内容,主要说明本论文的研究目的、内容、方法、成果和结论。要突出本论文的创造性成果或新见解,不要与引言相 混淆。语言力求精练、准确,以 300—500 字为宜。
% 在摘要的下方另起一行,注明本文的关键词(3—5 个)。关键词是供检索用的主题词条,应采用能覆盖论文主要内容的通用技术词条(参照相应的技术术语 标准)。按词条的外延层次排列(外延大的排在前面)。摘要与关键词应在同一页。
% modifier: 黄俊杰(huangjj27, 349373001dc@gmail.com)
% update date: 2017-04-15
%%

\cabstract{

    随着单精度甚至更低精度浮点格式在现代AI应用中的普及,一个新基准HPL-AI被提出,以用于评估计算系统在混合精度计算场景上的性能。然而,该基准目前尚未有公开的通用实现。在本文中,我们实现了HPL-AI基准的首个开源的分布式实现,并在X86、CUDA、ARM64等多个平台对其正确性、通用性、可扩展性进行了测试。同时,我们将该实现移植到华为Atlas 800-9000训练服务器上,并针对其搭载的国产异构处理器华为昇腾910进行优化。实验结果表明,借助混合精度矩阵分解算法和数值迭代算法,我们的实现能获得与HPL相当甚至更高的计算精度,且带来显著性能提升。

}

% 中文关键词(每个关键词之间用“,”分开,最后一个关键词不打标点符号。)

\ckeywords{HPL-AI,MPI,混合精度,矩阵分解,数值迭代,华为昇腾910}

\eabstract{
    % 英文摘要及关键词内容应与中文摘要及关键词内容相同。中英文摘要及其关键词各置一页内。

    With the popularity of 32-bit and even lower floating-point precision formats in modern AI applications, a new benchmark, HPL-AI, was proposed to evaluate the performance of computing systems in mixed-precision scenarios. However, the benchmark currently had no publicly available general implementation. In this thesis, we completed the first open-source general MPI implementation of HPL-AI benchmark, and tested its correctness, versatility, and scalability on multiple platforms including X86, CUDA, and ARM64. Meanwhile, we transplanted the implementation to the HUAWEI Atlas 800-9000 training server, and optimized for domestic heterogeneous processors HUAWEI Ascend 910 on the server. Experimental results showed that with the help of the mixed-precision matrix factorization algorithm and iterative refinement algorithm, our implementation could obtain the accuracy equal to or even higher than that of HPL with a significant performance improvement.

}

% 英文文关键词(每个关键词之间用,分开, 最后一个关键词不打标点符号。)

\ekeywords{HPL-AI, MPI, mixed-precision, matrix factorization, iterative refinement, HUAWEI Ascend 910}
