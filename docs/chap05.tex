\chapter{总结与展望}
\label{总结与展望}

\section{总结}

在本文中,我们首先介绍了HPL-AI基准的背景和相关工作,并分析了HPL-AI的基本规则,介绍了使用到的矩阵生成算法、LU分解算法、数值迭代算法。

随后,基于对硬件架构的分析以及对接口、算子调用的Profiling,我们面向华为Atlas 800-9000 AI训练服务器上搭载的国产异构处理器昇腾910设计出一套合理的移植优化方案,以充分利用其半精度算力。

然后,我们以X86+CUDA平台为对照组,将HPL和我们的HPL-AI性能进行对比。实验结果表明,借助混合精度矩阵分解算法和数值迭代算法,我们的实现能获得与HPL相当的计算精度,且得到明显性能提升。

我们的工作为不同平台上的HPL-AI基准测试给出可信参照,也对在昇腾平台或其他异构计算平台上的HPC/AI应用优化有很强的指导意义。

\section{展望}

时间所限,在本次毕业设计前我仅完成了对HPL-AI的初步实现和移植。后续交付鹏城实验室后,以下工作仍然是值得被考虑的:

\begin{enumerate}
    \item 使用昇腾910的RC模式\footnote{\url{https://support.huaweicloud.com/asdevg-c-cann/atlassample_14_0010.html}}启动HPL-AI,免去通信开销。由于其板载CPU性能远低于主机端CPU性能,且内存只有16GB,实际表现仍是值得商榷的。
    \item 可以参照英伟达 hpl-2.0\_FERMI\_v15.tgz \footnote{\url{https://developer.nvidia.com/rdp/assets/cuda-accelerated-linpack-linux64}}中的手段,使用多流等手段提高异构处理器的负载;
    \item 由于CPU同为ARM架构,\cite{DBLP:conf/scala-ws/KudoNII20}在``富岳''上的工作对我们有很强的参考意义。例如:
          \begin{enumerate}
              \item Single iteration scheme for IR相较于GMRES开销更低;
              \item On-The-Fly matrix generation可减少主机端内存上的占用。当然,对于我们的移植平台,限制可能更多来自于异构处理器的存储空间。
          \end{enumerate}
    \item 探索传统HPL的优化方法,如针对``鹏城云脑II''的网络拓扑,优化HPL-AI的MPI调用。由于我们的HPL-AI是参照HPL实现的,大部分对于HPL的优化都可以考虑应用到HPL-AI上。
\end{enumerate}
