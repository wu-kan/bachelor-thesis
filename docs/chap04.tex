\chapter{实验及分析}
\label{实验及分析}

\section{硬件环境}
\label{硬件环境}

本节将介绍用于移植测试的华为Atlas 800-9000训练服务器和X86+CUDA对照组平台硬件配置。

\subsection{华为Atlas 800-9000训练服务器简介}
\label{华为Atlas800-9000训练服务器简介}

\newcommand{\tabincell}[2]{\begin{tabular}{@{}#1@{}}#2\end{tabular}}

结合产品主页\footnote{\url{https://e.huawei.com/en/products/cloud-computing-dc/atlas/atlas-800-training-9000}}和本地运行 \lstinline{dmidecode} 指令的结果,我们得到\autoref{华为Atlas800-9000主要配置表}。由于是在单机上进行的算力测试,此处省略了网络和存储等无关参数。

\begin{table}[h] %voc table result
    \centering
    \caption{华为Atlas800-9000主要配置表}
    \label{华为Atlas800-9000主要配置表}
    \resizebox{\textwidth}{!}{
        \begin{tabular}{c|c|c|c}
            \toprule
            组件 & 数量             & 型号         & 参数                          \\
            \midrule
            CPU  & 4                & 华为鲲鹏920  & \tabincell{c}{48Cores @2.6GHz \\L1 Cache 6144KB\\L2 Cache 24567KB\\L3 Cache 49152 kB} \\
            \hline
            内存 & \tabincell{c}{24                                                \\(4 Socket$\times$\\6 Channel)} & 镁光 36ASF4G72PZ-2G9E2 & \tabincell{c}{DDR4\\32GB\\2933 MT/s\\1.2V}\\
            \hline
            NPU  & 8                & 华为昇腾910A & \tabincell{c}{架构:达芬奇    \\FP16算力:320 TFLOPS\\HBM: 32GB, 1200GB/s} \\
            \bottomrule
        \end{tabular}}
\end{table}

容易看出,这台华为Atlas 800-9000是一台高性能服务器,有高达192个ARM64 CPU核,内存总容量共计768GB。同时,其上搭载了8块特殊的``NPU''(neural-network processing units,神经网络处理器),已在\autoref{华为昇腾910硬件架构}详细分析。

\subsection{X86+CUDA对照平台简介}

结合结合产品主页\footnote{\url{https://www.nvidia.com/en-us/data-center/a100/}}和本地运行 \lstinline{dmidecode} 指令的结果,我们得到\autoref{X86+CUDA对照组平台主要配置表}。由于是在单机上进行的算力测试,此处省略了网络和存储等无关参数。

\begin{table}[h] %voc table result
    \centering
    \caption{X86+CUDA对照组平台主要配置表}
    \label{X86+CUDA对照组平台主要配置表}
    \resizebox{\textwidth}{!}{
        \begin{tabular}{c|c|c|c}
            \toprule
            组件 & 数量             & 型号                  & 参数                                    \\
            \midrule
            CPU  & 2                & AMD 霄龙 7542         & \tabincell{c}{32Cores, 64Threads@2.9GHz \\L1 Cache 2MB\\L2 Cache 16MB\\L3 Cache 128 MB} \\
            \hline
            内存 & \tabincell{c}{32                                                                   \\(4 Socket$\times$\\8 Channel)} & 海力士 HMAA8GR7AJR4N-XN & \tabincell{c}{DDR4\\64GB\\2933 MT/s\\1.2V}\\
            \hline
            GPU  & 8                & 英伟达 A100-SXM4-40GB & \tabincell{c}{架构:安培                \\FP16算力:312 TFLOPS(TensorCore)\\
                FP32算力:156 TFLOPS(TensorCore)                                                    \\
                HBM: 40GB, 1555GB/s}                                                                  \\
            \bottomrule
        \end{tabular}}
\end{table}

该平台作为对照组的优势是,其拥有与移植平台相近的配置,且搭载的加速卡拥有相同的7nm制程、相近的FP16理论矩阵算力。当然,仍然要指出的是,华为Atlas 800-9000所搭载的鲲鹏920与对照组AMD霄龙7542的架构完全不同,前者是ARM64,后者是X86;并且,前者拥有更多的核心,后者有更高的频率和更大的Cache。

\section{软件环境}

\autoref{测试平台软件环境表}给出了我们的实验环境,表中软件除了针对昇腾异构加速卡特殊优化的hpl-ai@2.3d+ascendcl及其依赖CANN V100R020C10外,均可通过包管理器spack\cite{2015The}与我们的spack软件扩展源一键安装。\footnote{我们使用了spack包管理器管理需要的诸多软件依赖,其优势在于提供了一套易用的源码安装管理。我们也将我们的HPL-AI发布在\url{https://github.com/SYSU-SCC/sysu-scc-spack-repo}中,这使得其可以很容易地在其他环境中被部署、复现。}

\begin{table}[h]
    \centering
    \caption{测试平台软件环境表}
    \label{测试平台软件环境表}
    \resizebox{\textwidth}{!}{

        \begin{tabular}{c|c|c}
            \toprule

            平台     & 华为Atlas 800-9000                            & X86+CUDA对照组平台        \\

            \midrule

            操作系统 & linux-centos7-aarch64                         & linux-debian9-zen2        \\

            \hline

            驱动版本 & ascend-dmi 20.1.rc1                           & NVIDIA-SMI 450.80.02      \\

            \hline

            特定软件 & \begin{tabular}[c]{@{}c@{}}
                CANN V100R020C10 \\
                hpl-ai@2.3d\%gcc@7.5.0+ascendcl
            \end{tabular}                     & \begin{tabular}[c]{@{}c@{}}
                cuda@11.0.2\%gcc@7.5.0            \\
                blaspp@2021.04.01\%gcc@7.5.0+cuda \\
                hpl-ai@2.3d+cuda                  \\
                hpl-ai@2.3d+cuda $\backslash$     \\
                cublasGemmEx\_computeType=fp16
            \end{tabular} \\

            \hline

            通用软件 & \multicolumn{2}{c}{\begin{tabular}[c]{@{}c@{}}
                    spack 0.16.1                              \\
                    gcc@7.5.0                                 \\
                    openmpi@3.1.6\%gcc@7.5.0                  \\
                    openblas@0.3.13\%gcc@7.5.0 threads=openmp \\
                    hpl@2.3\%gcc@7.5.0                        \\
                    blaspp@2021.04.01\%gcc@7.5.0              \\
                    hpl-ai@2.3d\%gcc@7.5.0                    \\
                \end{tabular}}                             \\

            \bottomrule
        \end{tabular}}
\end{table}

\subsection{对比软件介绍}

\autoref{对比软件介绍表}列出了我们实现的HPL-AI软件包的和用于对照组的开源HPL软件包。

\begin{table}[h] %voc table result
    \centering
    \caption{对比软件介绍表}
    \label{对比软件介绍表}
    \resizebox{\textwidth}{!}{
        \begin{tabular}{c|c|c}
            \toprule
            软件                 & 运行设备 & 介绍                             \\
            \midrule
            hpl@2.3              & CPU      & \tabincell{c}{HPL标准实现。      \\用于对照组。}\\
            \hline
            hpl-ai@2.3d          & CPU      & \tabincell{c}{我们的HPL-AI实现。 \\在LU分解阶段使用单精度。}\\
            \hline
            hpl-ai@2.3d+ascendcl & CPU+NPU  & \tabincell{c}{
                我们的HPL-AI实现。                                             \\
                在LU分解阶段使用单精度。                                       \\
                GEMM阶段会在NPU上将FP32格式的矩阵继续压缩成FP16格式,          \\
                然后调用MatMulV2。}                                            \\
            \hline
            \tabincell{c}{
            hpl-ai@2.3d+cuda}    & CPU+GPU  & \tabincell{c}{
                我们的HPL-AI实现。                                             \\
                在LU分解阶段使用单精度。                                       \\
                使用GPU完成gemm和trsm过程。}                                   \\
            \hline
            \tabincell{c}{
                hpl-ai@2.3d+cuda $\backslash$                                  \\
                cublasGemmEx\_computeType=fp16
            }                    & CPU+GPU  & \tabincell{c}{
                我们的HPL-AI实现。                                             \\
                在LU分解阶段使用单精度。                                       \\
                gemm阶段会在GPU上调用cublasGemmEx,                            \\
                将FP32格式的矩阵继续压缩成FP16格式并进行矩阵乘法。}            \\
            \bottomrule
        \end{tabular}}
\end{table}

\subsection{软件依赖}
\label{软件依赖}

CPU厂商通常会针对自己的硬件发布经过调优的BLAS实现,如Intel的mkl库、AMD的amdblis库等(这些厂商甚至会发布针对自己硬件优化的Linpack实现,如Intel的mp\_linpack、NVIDIA的hpc-benchmark),但这为我们在不同平台上进行性能比较带来了不便。一方面,这些调优后的数学库大部分情况下都不能移植到其他平台(或是性能表现很差),难以控制实验中的变量;另一方面,其实现通常不开源,这使得我们更难分析运行时的瓶颈是来自于算法本身还是调用的BLAS。为此,我们在测试时使用openblas@0.3.13\footnote{需要注意的是,spack 0.16.1内建的openblas@0.3.5存在问题,详见\autoref{运行策略}。我们在自己的spack软件扩展源提供了修复该错误的openblas@0.3.13及改变软件依赖的blaspp安装脚本。},该库提供了X86、ARM、RISC-V等多种CPU平台的开源实现。

同理,我们选择了openmpi@3.1.6作为作为统一的MPI依赖,并且所有提供源码的软件包统一使用gcc@7.5.0从源码编译。

\subsection{MPI进程及绑核设置}

\autoref{硬件环境}中提到,不管是华为Atlas 800-9000还是用于对照的X86+CUDA平台,都有八张加速卡。我们让单张加速卡对应一个单独进程,因此如果没有特别提及的情况下统一使用八个MPI进程。同时,使用openmpi@3.1.6默认的绑核设置。

此外,由于下述实验都是在单机上进行的,因此不涉及MPI通信协议的选择。

\subsection{OpenMP线程数设置}

即使是GPU/NPU上运行的HPL-AI,都有相当一部分计算量(主要是level-1和level-2的BLAS调用)是放在CPU上的,因此需要设置openblas库运行时用于单进程的线程数量。设置的原则是充分使用CPU的每个计算核心。对于Atlas,我们设置8个进程每个进程24线程;对于X86+CUDA对照平台,我们设置8个进程每个进程16线程。

\subsection{HPL.dat}

运行参数的设置是一个稍微复杂一些的问题。一方面,作为benchmark,它需要尽量契合硬件配置,以充分利用机器能提供的最高算力;另一方面,在对照实验中,我们应当控制变量,从而尽可能排除由于参数设置不同影响实验结果的情况。因此,有必要使用一个尽可能通用的运行参数设置,同时针对各个实验进行微调。如果没有特别提及的输入参数,均默认使用\autoref{HPL.dat}中的内容。

\begin{figure}[h]
    \centering
    \begin{bash}
        HPLinpack and HPL-AI benchmark input file
        National Supercomputer Center in Guangzhou, Sun Yat-sen University
        HPL.out      output file name (if any)
        6            device out (6=stdout,7=stderr,file)
        1            # of problems sizes (N)
        131072       Ns
        2            # of NBs
        192 192      NBs
        0            PMAP process mapping (0=Row-,1=Column-major)
        1            # of process grids (P x Q)
        2            Ps
        4            Qs
        16.0         threshold
        1            # of panel fact
        2 1 0        PFACTs (0=left, 1=Crout, 2=Right)
        1            # of recursive stopping criterium
        2 8          NBMINs (>= 1)
        1            # of panels in recursion
        2            NDIVs
        1            # of recursive panel fact.
        2 1 0        RFACTs (0=left, 1=Crout, 2=Right)
        1            # of broadcast
        0 2          BCASTs (0=1rg,1=1rM,2=2rg,3=2rM,4=Lng,5=LnM)
        1            # of lookahead depth
        1            DEPTHs (>=0)
        1            SWAP (0=bin-exch,1=long,2=mix)
        192          swapping threshold
        1            L1 in (0=transposed,1=no-transposed) form
        0            U  in (0=transposed,1=no-transposed) form
        1            Equilibration (0=no,1=yes)
        16           memory alignment in HPLAI_T_AFLOAT (> 0)
    \end{bash}
    \caption{HPL.dat}
    \label{HPL.dat}
\end{figure}

此处重复设置$\mathit{NB}$以使用\autoref{热身运行}中的策略。值得注意的是,此处设置$\mathit{NB}=192$,这是适合于CPU的参数,以减少Cache Miss。在异构处理器上运行HPL或HPL-AI时通常需要取更大的值,从而充分利用异构处理器的矩阵算力。在本文中对于NPU我们选取$\mathit{NB}=4096$,对于GPU我们选取$\mathit{NB}=2048$。

\section{实验与分析}

我们设计了如下三个实验,对我们的HPL-AI软件包进行测试。

\subsection{AscendCL接口与算子调用实验}
\label{AscendCL接口与算子调用实验}

运行我们的Profile代码\footnote{我们的完整Profile代码提供在\url{https://wu-kan.cn/2021/03/21/\%E5\%8D\%8E\%E4\%B8\%BA\%E6\%98\%87\%E8\%85\%BE910\%E4\%BD\%BF\%E7\%94\%A8\%E5\%88\%9D\%E6\%8E\%A2/}}得到\autoref{AscendCL接口/算子耗时表},给出$\left(M,N,K\right)=\left(8192,8192,8192\right)$时各接口和算子的耗时情况。

\begin{table}[h] %voc table result
    \centering
    \caption{AscendCL接口/算子耗时表}
    \label{AscendCL接口/算子耗时表}
    \begin{tabular}{c|c|c}
        \toprule
        接口/算子              & 运行时间    & 有效算力       \\
        \midrule
        aclInit                & 4.422 ms    & /              \\
        \hline
        aclrtCreateContext     & 351.026 ms  & /              \\
        \hline
        aclrtSetCurrentContext & 0.006 ms    & /              \\
        \hline
        aclrtMalloc            & 198.336 ms  & /              \\
        \hline
        aclrtMemcpyAsync       & 11.689 ms   & /              \\
        \hline
        Cast(aclFloat16)       & 0.498 ms    & /              \\
        \hline
        Cast(float)            & 0.484 ms    & /              \\
        \hline
        MatMulV2(FP16)         & 5.613 ms    & 195.887 TFLOPS \\
        \hline
        MatMulV2(FP32)         & 1154.355 ms & 0.952 TFLOPS   \\
        \hline
        MatMul(FP16)           & 5.610 ms    & 195.991 TFLOPS \\
        \hline
        MatMul(FP32)           & 1154.473 ms & 0.952 TFLOPS   \\
        \hline
        GEMM(FP16)             & 202.152 ms  & 5.439 TFLOPS   \\
        \hline
        aclrtFree              & 59.107 ms   & /              \\
        \hline
        aclrtDestroyContext    & 186.624 ms  & /              \\
        \hline
        aclFinalize            & 349.656 ms  & /              \\
        \hline
        \bottomrule
    \end{tabular}
\end{table}

根据 Profile 结果可以得到以下结论:

\begin{itemize}
    \item 创建上下文、申请内存、释放内存、释放上下文、去初始化的开销都很大,需要在运行算法时尽量避免相关调用。
    \item \lstinline{MatMul} 算子调用时间远低于三次 \lstinline{aclrtMemcpy}(表中Memcpy仅包含一个矩阵拷贝的时间)。换言之,开销主要在主机和设备的通信,而非设备上的计算上。
    \item 切换上下文开销很低,但创建释放上下文的开销仍然很高,必要时仍要使用 \lstinline{aclrtStream}。
    \item \lstinline{GEMM} 算子性能远低于 \lstinline{MatMul} 和 \lstinline{MatMulV2},合理推断向量单元和标量单元性能远低于矩阵单元。
    \item 使用单精度的MatMul比半精度的MatMul慢了几个数量级,比较鸡肋。
\end{itemize}

\subsection{可扩展性验证实验}

对于$N\in\lbrace 4096,8192,16384,32768,65536,131072\rbrace$,在Atlas上使用8、16、32、64、128、192核心运行hpl-ai@2.3d,得到\autoref{HPL-AI效率测试表}(/TFLOPS),验证HPL-AI软件包的可扩展性。

\begin{table}[h] %voc table result
    \centering
    \caption{HPL-AI效率测试表}
    \label{HPL-AI效率测试表}
    \begin{tabular}{c|cccccc}
        \toprule
        问题规模$\backslash$并行核数 & 8     & 16    & 32    & 64    & 128   & 192   \\
        \midrule
        4096                         & 0.013 & 0.215 & 0.323 & 0.428 & 0.456 & 0.112 \\
        \hline
        8192                         & 0.016 & 0.305 & 0.526 & 0.816 & 1.052 & 1.080 \\
        \hline
        16384                        & 0.019 & 0.358 & 0.673 & 1.172 & 1.683 & 2.018 \\
        \hline
        32768                        & 0.020 & 0.389 & 0.752 & 1.331 & 2.331 & 2.766 \\
        \hline
        65536                        & 0.021 & 0.417 & 0.816 & 1.532 & 2.732 & 3.660 \\
        \hline
        131072                       & 0.022 & 0.432 & 0.853 & 1.660 & 3.103 & 4.340 \\
        \bottomrule
    \end{tabular}
\end{table}

由\autoref{HPL-AI具有强可扩展性}可以看到,问题规模足够大时,固定问题(不妨取$N=131072$)规模,有效算力随并行核数增加线性增长,该应用有\textbf{强可扩展性}。在问题规模小的时候(如$N=4096$),单节点过多的并行核数反而会导致诸如伪共享\footnote{\url{https://en.wikipedia.org/wiki/False_sharing}}的问题,使程序变慢。

\begin{figure}[h]
    \centering
    \begin{tikzpicture}
        \begin{axis}[xlabel=并行核数,ylabel=效率/TFLOPS,legend pos=outer north east] % 将图例放在图外,位于图的东北角
            \addplot
            coordinates                             % 绘制原始数据的折线图
                {           		                % X,Y的原始数据
                    (8,0.22)
                    (16,0.432)
                    (32,0.853)
                    (64,1.660)
                    (128,3.103)
                    (192,4.340)
                };
            \addlegendentry{$N=131072$}
            \addplot
            coordinates                              % 绘制原始数据的折线图
                {           		                % X,Y的原始数据
                    (8,0.21)
                    (16,0.417)
                    (32,0.816)
                    (64,1.532)
                    (128,2.732)
                    (192,3.660)
                };
            \addlegendentry{$N=65536$}
            \addplot
            coordinates                                 % 绘制原始数据的折线图
                {           		                % X,Y的原始数据
                    (8,0.020)
                    (16,0.389)
                    (32,0.752)
                    (64,1.331)
                    (128,2.331)
                    (192,2.766)
                };
            \addlegendentry{$N=32768$}
            \addplot
            coordinates                                 % 绘制原始数据的折线图
                {           		                % X,Y的原始数据
                    (8,0.019)
                    (16,0.358)
                    (32,0.673)
                    (64,1.172)
                    (128,1.683)
                    (192,2.018)
                };
            \addlegendentry{$N=16384$}
            \addplot
            coordinates                                 % 绘制原始数据的折线图
                {           		                % X,Y的原始数据
                    (8,0.016)
                    (16,0.305)
                    (32,0.526)
                    (64,0.816)
                    (128,1.052)
                    (192,1.080)
                };
            \addlegendentry{$N=8192$}
            \addplot
            coordinates                               % 绘制原始数据的折线图
                {           		                % X,Y的原始数据
                    (8,0.013)
                    (16,0.215)
                    (32,0.323)
                    (64,0.428)
                    (128,0.456)
                    (192,0.112)
                };
            \addlegendentry{$N=4096$}
        \end{axis}
    \end{tikzpicture}
    \caption{HPL-AI具有强可扩展性}
    \label{HPL-AI具有强可扩展性}
\end{figure}

\subsection{NPU加速验证实验}

在Atlas上运行hpl@2.3、hpl-ai@2.3d、hpl-ai@2.3d+ascendcl($\mathit{NB}=4096$),在X86+CUDA验证平台运行hpl@2.3、hpl-ai@2.3d、hpl-ai@2.3d+cuda($\mathit{NB}=2048$)、hpl-ai@2.3d+cuda cublasGemmEx\_computeType=fp16($\mathit{NB}=2048$),对比并验证hpl-ai@2.3d+ascendcl的移植加速效果,得到\autoref{NPU加速验证表}和\autoref{实测性能对照图}。

\begin{table}[h] %voc table result
    \centering
    \caption{NPU加速验证表}
    \label{NPU加速验证表}
    \resizebox{\textwidth}{!}{
        \begin{tabular}{c|c|c|c|c}
            \toprule
            编号                  & 测试软件包                                      & 测试平台 & 运行时间      & 有效算力     \\
            \midrule
            \#1                   & hpl@2.3                                         & Atlas    & 905.95s       & 1.657 TFLOPS \\
            \hline
            \#2                   & hpl-ai@2.3d                                     & Atlas    & 345.90s       & 4.340 TFLOPS \\
            \hline
            \#3                   & \tabincell{c}{hpl-ai@2.3d+ascendcl $\backslash$                                           \\($\mathit{NB}=4096$)} & Atlas & 261.84s & 5.733 TFLOPS  \\
            \hline
            \#4                   & hpl@2.3                                         & X86+CUDA & 718.88s       & 2.088 TFLOPS \\
            \hline
            \#5                   & hpl-ai@2.3d                                     & X86+CUDA & 360.70s       & 4.162 TFLOPS \\
            \hline
            \#6                   & \tabincell{c}{
                hpl-ai@2.3d+cuda $\backslash$                                                                                 \\
            ($\mathit{NB}=2048$)} & X86+CUDA                                        & 112.95s  & 13.291 TFLOPS                \\
            \hline
            \#7                   & \tabincell{c}{
                hpl-ai@2.3d+cuda $\backslash$                                                                                 \\
                cublasGemmEx\_computeType=fp16 $\backslash$                                                                   \\($\mathit{NB}=2048$)} & X86+CUDA & 103.69s & 14.477 TFLOPS  \\
            \bottomrule
        \end{tabular}}

\end{table}

\begin{figure}[h]
    \centering
    \begin{tikzpicture}
        \begin{axis}
            [
                ybar,
                ylabel={实测性能/TFLOPS}, % the ylabel must precede a # symbol.
                xlabel={实验算例},
                symbolic x coords={\#1, \#2, \#3, \#4, \#5, \#6, \#7},
                legend pos=outer north east
            ]

            \addplot coordinates {(\#1, 1.657) (\#2, 4.296) (\#3, 5.733)};
            \addlegendentry{ARM+NPU};
            \addplot coordinates {(\#4, 2.088) (\#5, 4.162) (\#6, 13.291) (\#7, 14.477)};
            \addlegendentry{X86+GPU};

        \end{axis}
    \end{tikzpicture}
    \caption{实测性能对照图}
    \label{实测性能对照图}
\end{figure}

对比\#1、\#2可知,相对于HPL,我们的HPL-AI在Atlas使用单精度取得了$2.59\times$的加速比。一方面,使用单精度运算可以使用比FP64运算单元更多的FP32计算单元;另一方面,使用低精度计算也减少了算法运行时的通信开销和CacheMiss。

对比\#5、\#6、\#7可知,我们的HPL-AI可以使用异构加速卡取得明显性能提升,且引入混合精度的计算可以获得更高的收益。

对比\#3、\#7,虽然此处的参数$N$设置虽然均远没有让两台机器的异构处理器满载\footnote{此处选择的$N$需要大量不同规格的Cast+MatMulV2算子,即使使用\autoref{算子编译脚本}中的并行编译脚本也需要数个小时才能进行一次完整测试。继续扩大$N$会导致算子数量的进一步上升,短时间内很难进行进行参数调优。},也可明显对比NPU与GPU的性能差异。

一方面,由\autoref{AscendCL接口与算子调用实验}可知,在我们的应用中使用昇腾910的瓶颈主要来自于三次\lstinline{aclrtMemcpy}过程,其耗时是调用\lstinline{MatMul}算子的六倍之多;而此处我们使用的A100是NVLink接口版本,其带宽高达600GB/S,远大于使用昇腾910的64GB/S(PCIe4.0)\footnote{\autoref{AscendCL接口与算子调用实验}中实测性能约为21GB/S},因此可以更加充分的利用异构加速卡上的算力。因此,通信带宽是昇腾910的一个明显瓶颈;若后续版本的昇腾处理器提升其与主机的通信带宽,则可在HPL-AI基准中获得更高的计算效率。

另一方面,昇腾910是专门面向为AI应用设计的处理器,本身并非适合通用计算,而NVIDIA A100则可以使用cuBLAS中专门调优过的gemm接口和trsm接口。国产异构处理器在软件生态建设上仍任重而道远。
